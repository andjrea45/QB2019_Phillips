\documentclass[]{article}
\usepackage{lmodern}
\usepackage{amssymb,amsmath}
\usepackage{ifxetex,ifluatex}
\usepackage{fixltx2e} % provides \textsubscript
\ifnum 0\ifxetex 1\fi\ifluatex 1\fi=0 % if pdftex
  \usepackage[T1]{fontenc}
  \usepackage[utf8]{inputenc}
\else % if luatex or xelatex
  \ifxetex
    \usepackage{mathspec}
  \else
    \usepackage{fontspec}
  \fi
  \defaultfontfeatures{Ligatures=TeX,Scale=MatchLowercase}
\fi
% use upquote if available, for straight quotes in verbatim environments
\IfFileExists{upquote.sty}{\usepackage{upquote}}{}
% use microtype if available
\IfFileExists{microtype.sty}{%
\usepackage{microtype}
\UseMicrotypeSet[protrusion]{basicmath} % disable protrusion for tt fonts
}{}
\usepackage[margin=2.54cm]{geometry}
\usepackage{hyperref}
\hypersetup{unicode=true,
            pdftitle={11. Worksheet: Phylogenetic Diversity - Traits},
            pdfauthor={Andrea Phillips; Z620: Quantitative Biodiversity, Indiana University},
            pdfborder={0 0 0},
            breaklinks=true}
\urlstyle{same}  % don't use monospace font for urls
\usepackage{color}
\usepackage{fancyvrb}
\newcommand{\VerbBar}{|}
\newcommand{\VERB}{\Verb[commandchars=\\\{\}]}
\DefineVerbatimEnvironment{Highlighting}{Verbatim}{commandchars=\\\{\}}
% Add ',fontsize=\small' for more characters per line
\usepackage{framed}
\definecolor{shadecolor}{RGB}{248,248,248}
\newenvironment{Shaded}{\begin{snugshade}}{\end{snugshade}}
\newcommand{\KeywordTok}[1]{\textcolor[rgb]{0.13,0.29,0.53}{\textbf{#1}}}
\newcommand{\DataTypeTok}[1]{\textcolor[rgb]{0.13,0.29,0.53}{#1}}
\newcommand{\DecValTok}[1]{\textcolor[rgb]{0.00,0.00,0.81}{#1}}
\newcommand{\BaseNTok}[1]{\textcolor[rgb]{0.00,0.00,0.81}{#1}}
\newcommand{\FloatTok}[1]{\textcolor[rgb]{0.00,0.00,0.81}{#1}}
\newcommand{\ConstantTok}[1]{\textcolor[rgb]{0.00,0.00,0.00}{#1}}
\newcommand{\CharTok}[1]{\textcolor[rgb]{0.31,0.60,0.02}{#1}}
\newcommand{\SpecialCharTok}[1]{\textcolor[rgb]{0.00,0.00,0.00}{#1}}
\newcommand{\StringTok}[1]{\textcolor[rgb]{0.31,0.60,0.02}{#1}}
\newcommand{\VerbatimStringTok}[1]{\textcolor[rgb]{0.31,0.60,0.02}{#1}}
\newcommand{\SpecialStringTok}[1]{\textcolor[rgb]{0.31,0.60,0.02}{#1}}
\newcommand{\ImportTok}[1]{#1}
\newcommand{\CommentTok}[1]{\textcolor[rgb]{0.56,0.35,0.01}{\textit{#1}}}
\newcommand{\DocumentationTok}[1]{\textcolor[rgb]{0.56,0.35,0.01}{\textbf{\textit{#1}}}}
\newcommand{\AnnotationTok}[1]{\textcolor[rgb]{0.56,0.35,0.01}{\textbf{\textit{#1}}}}
\newcommand{\CommentVarTok}[1]{\textcolor[rgb]{0.56,0.35,0.01}{\textbf{\textit{#1}}}}
\newcommand{\OtherTok}[1]{\textcolor[rgb]{0.56,0.35,0.01}{#1}}
\newcommand{\FunctionTok}[1]{\textcolor[rgb]{0.00,0.00,0.00}{#1}}
\newcommand{\VariableTok}[1]{\textcolor[rgb]{0.00,0.00,0.00}{#1}}
\newcommand{\ControlFlowTok}[1]{\textcolor[rgb]{0.13,0.29,0.53}{\textbf{#1}}}
\newcommand{\OperatorTok}[1]{\textcolor[rgb]{0.81,0.36,0.00}{\textbf{#1}}}
\newcommand{\BuiltInTok}[1]{#1}
\newcommand{\ExtensionTok}[1]{#1}
\newcommand{\PreprocessorTok}[1]{\textcolor[rgb]{0.56,0.35,0.01}{\textit{#1}}}
\newcommand{\AttributeTok}[1]{\textcolor[rgb]{0.77,0.63,0.00}{#1}}
\newcommand{\RegionMarkerTok}[1]{#1}
\newcommand{\InformationTok}[1]{\textcolor[rgb]{0.56,0.35,0.01}{\textbf{\textit{#1}}}}
\newcommand{\WarningTok}[1]{\textcolor[rgb]{0.56,0.35,0.01}{\textbf{\textit{#1}}}}
\newcommand{\AlertTok}[1]{\textcolor[rgb]{0.94,0.16,0.16}{#1}}
\newcommand{\ErrorTok}[1]{\textcolor[rgb]{0.64,0.00,0.00}{\textbf{#1}}}
\newcommand{\NormalTok}[1]{#1}
\usepackage{graphicx,grffile}
\makeatletter
\def\maxwidth{\ifdim\Gin@nat@width>\linewidth\linewidth\else\Gin@nat@width\fi}
\def\maxheight{\ifdim\Gin@nat@height>\textheight\textheight\else\Gin@nat@height\fi}
\makeatother
% Scale images if necessary, so that they will not overflow the page
% margins by default, and it is still possible to overwrite the defaults
% using explicit options in \includegraphics[width, height, ...]{}
\setkeys{Gin}{width=\maxwidth,height=\maxheight,keepaspectratio}
\IfFileExists{parskip.sty}{%
\usepackage{parskip}
}{% else
\setlength{\parindent}{0pt}
\setlength{\parskip}{6pt plus 2pt minus 1pt}
}
\setlength{\emergencystretch}{3em}  % prevent overfull lines
\providecommand{\tightlist}{%
  \setlength{\itemsep}{0pt}\setlength{\parskip}{0pt}}
\setcounter{secnumdepth}{0}
% Redefines (sub)paragraphs to behave more like sections
\ifx\paragraph\undefined\else
\let\oldparagraph\paragraph
\renewcommand{\paragraph}[1]{\oldparagraph{#1}\mbox{}}
\fi
\ifx\subparagraph\undefined\else
\let\oldsubparagraph\subparagraph
\renewcommand{\subparagraph}[1]{\oldsubparagraph{#1}\mbox{}}
\fi

%%% Use protect on footnotes to avoid problems with footnotes in titles
\let\rmarkdownfootnote\footnote%
\def\footnote{\protect\rmarkdownfootnote}

%%% Change title format to be more compact
\usepackage{titling}

% Create subtitle command for use in maketitle
\newcommand{\subtitle}[1]{
  \posttitle{
    \begin{center}\large#1\end{center}
    }
}

\setlength{\droptitle}{-2em}

  \title{11. Worksheet: Phylogenetic Diversity - Traits}
    \pretitle{\vspace{\droptitle}\centering\huge}
  \posttitle{\par}
    \author{Andrea Phillips; Z620: Quantitative Biodiversity, Indiana University}
    \preauthor{\centering\large\emph}
  \postauthor{\par}
      \predate{\centering\large\emph}
  \postdate{\par}
    \date{19 February, 2019}


\begin{document}
\maketitle

\subsection{OVERVIEW}\label{overview}

Up to this point, we have been focusing on patterns taxonomic diversity
in Quantitative Biodiversity. Although taxonomic diversity is an
important dimension of biodiversity, it is often necessary to consider
the evolutionary history or relatedness of species. The goal of this
exercise is to introduce basic concepts of phylogenetic diversity.

After completing this exercise you will be able to:

\begin{enumerate}
\def\labelenumi{\arabic{enumi}.}
\tightlist
\item
  create phylogenetic trees to view evolutionary relationships from
  sequence data
\item
  map functional traits onto phylogenetic trees to visualize the
  distribution of traits with respect to evolutionary history
\item
  test for phylogenetic signal within trait distributions and
  trait-based patterns of biodiversity
\end{enumerate}

\subsection{Directions:}\label{directions}

\begin{enumerate}
\def\labelenumi{\arabic{enumi}.}
\tightlist
\item
  In the Markdown version of this document in your cloned repo, change
  ``Student Name'' on line 3 (above) with your name.
\item
  Complete as much of the worksheet as possible during class.
\item
  Use the handout as a guide; it contains a more complete description of
  data sets along with examples of proper scripting needed to carry out
  the exercises.
\item
  Answer questions in the worksheet. Space for your answers is provided
  in this document and is indicated by the ``\textgreater{}'' character.
  If you need a second paragraph be sure to start the first line with
  ``\textgreater{}''. You should notice that the answer is highlighted
  in green by RStudio (color may vary if you changed the editor theme).
\item
  Before you leave the classroom today, it is \emph{imperative} that you
  \textbf{push} this file to your GitHub repo, at whatever stage you
  are. This will enable you to pull your work onto your own computer.
\item
  When you have completed the worksheet, \textbf{Knit} the text and code
  into a single PDF file by pressing the \texttt{Knit} button in the
  RStudio scripting panel. This will save the PDF output in your
  `8.BetaDiversity' folder.
\item
  After Knitting, please submit the worksheet by making a \textbf{push}
  to your GitHub repo and then create a \textbf{pull request} via
  GitHub. Your pull request should include this file
  (\textbf{11.PhyloTraits\_Worksheet.Rmd}) with all code blocks filled
  out and questions answered) and the PDF output of \texttt{Knitr}
  (\textbf{11.PhyloTraits\_Worksheet.pdf}).
\end{enumerate}

The completed exercise is due on \textbf{Wednesday, February
20\textsuperscript{th}, 2019 before 12:00 PM (noon)}.

\subsection{1) SETUP}\label{setup}

Typically, the first thing you will do in either an R script or an
RMarkdown file is setup your environment. This includes things such as
setting the working directory and loading any packages that you will
need.

In the R code chunk below, provide the code to:\\
1. clear your R environment,\\
2. print your current working directory,\\
3. set your working directory to your ``\emph{/11.PhyloTraits}'' folder,
and\\
4. load all of the required R packages (be sure to install if needed).

\begin{Shaded}
\begin{Highlighting}[]
\KeywordTok{rm}\NormalTok{(}\DataTypeTok{list =} \KeywordTok{ls}\NormalTok{())}
\KeywordTok{getwd}\NormalTok{()}
\end{Highlighting}
\end{Shaded}

\begin{verbatim}
## [1] "C:/Users/andjr/Github2/QB2019_Phillips/2.Worksheets/11.PhyloTraits"
\end{verbatim}

\begin{Shaded}
\begin{Highlighting}[]
\CommentTok{#setwd("~/GitHub/QB-2019/2.Worksheets/11.PhyloTraits")}
\end{Highlighting}
\end{Shaded}

\subsection{2) DESCRIPTION OF DATA}\label{description-of-data}

The maintenance of biodiversity is thought to be influenced by
\textbf{trade-offs} among species in certain functional traits. One such
trade-off involves the ability of a highly specialized species to
perform exceptionally well on a particular resource compared to the
performance of a generalist. In this exercise, we will take a
phylogenetic approach to mapping phosphorus resource use onto a
phylogenetic tree while testing for specialist-generalist trade-offs.

\subsection{3) SEQUENCE ALIGNMENT}\label{sequence-alignment}

\textbf{\emph{Question 1}}: Using your favorite text editor, compare the
\texttt{p.isolates.fasta} file and the \texttt{p.isolates.afa} file.
Describe the differences that you observe between the two files.

\begin{quote}
\textbf{\emph{Answer 1}}:
\end{quote}

In the R code chunk below, do the following: 1. read your alignment
file, 2. convert the alignment to a DNAbin object, 3. select a region of
the gene to visualize (try various regions), and 4. plot the alignment
using a grid to visualize rows of sequences.

\begin{Shaded}
\begin{Highlighting}[]
\NormalTok{package.list <-}\StringTok{ }\KeywordTok{c}\NormalTok{(}\StringTok{'ape'}\NormalTok{, }\StringTok{'seqinr'}\NormalTok{, }\StringTok{'phylobase'}\NormalTok{, }\StringTok{'adephylo'}\NormalTok{, }\StringTok{'geiger'}\NormalTok{, }\StringTok{'picante'}\NormalTok{, }\StringTok{'stats'}\NormalTok{, }\StringTok{'RColorBrewer'}\NormalTok{, }\StringTok{'caper'}\NormalTok{, }\StringTok{'phylolm'}\NormalTok{, }\StringTok{'pmc'}\NormalTok{, }\StringTok{'ggplot2'}\NormalTok{, }\StringTok{'tidyr'}\NormalTok{, }\StringTok{'dplyr'}\NormalTok{, }\StringTok{'phangorn'}\NormalTok{, }\StringTok{'pander'}\NormalTok{) }
\ControlFlowTok{for}\NormalTok{ (package }\ControlFlowTok{in}\NormalTok{ package.list) \{}
  \ControlFlowTok{if}\NormalTok{ (}\OperatorTok{!}\KeywordTok{require}\NormalTok{(package, }\DataTypeTok{character.only=}\OtherTok{TRUE}\NormalTok{, }\DataTypeTok{quietly=}\OtherTok{TRUE}\NormalTok{)) \{}
    \KeywordTok{install.packages}\NormalTok{(package)}
    \KeywordTok{library}\NormalTok{(package, }\DataTypeTok{character.only=}\OtherTok{TRUE}\NormalTok{)}
\NormalTok{  \}}
\NormalTok{\}}
\end{Highlighting}
\end{Shaded}

\begin{verbatim}
## 
## Attaching package: 'seqinr'
\end{verbatim}

\begin{verbatim}
## The following objects are masked from 'package:ape':
## 
##     as.alignment, consensus
\end{verbatim}

\begin{verbatim}
## 
## Attaching package: 'phylobase'
\end{verbatim}

\begin{verbatim}
## The following object is masked from 'package:ape':
## 
##     edges
\end{verbatim}

\begin{verbatim}
## 
## Attaching package: 'permute'
\end{verbatim}

\begin{verbatim}
## The following object is masked from 'package:seqinr':
## 
##     getType
\end{verbatim}

\begin{verbatim}
## This is vegan 2.5-4
\end{verbatim}

\begin{verbatim}
## 
## Attaching package: 'nlme'
\end{verbatim}

\begin{verbatim}
## The following object is masked from 'package:seqinr':
## 
##     gls
\end{verbatim}

\begin{verbatim}
## 
## Attaching package: 'dplyr'
\end{verbatim}

\begin{verbatim}
## The following object is masked from 'package:MASS':
## 
##     select
\end{verbatim}

\begin{verbatim}
## The following object is masked from 'package:nlme':
## 
##     collapse
\end{verbatim}

\begin{verbatim}
## The following object is masked from 'package:seqinr':
## 
##     count
\end{verbatim}

\begin{verbatim}
## The following objects are masked from 'package:stats':
## 
##     filter, lag
\end{verbatim}

\begin{verbatim}
## The following objects are masked from 'package:base':
## 
##     intersect, setdiff, setequal, union
\end{verbatim}

\begin{verbatim}
## 
## Attaching package: 'phangorn'
\end{verbatim}

\begin{verbatim}
## The following objects are masked from 'package:vegan':
## 
##     diversity, treedist
\end{verbatim}

\begin{Shaded}
\begin{Highlighting}[]
\CommentTok{#muscle -in ./data/p.isolates.fasta -out ./data/p.isolates.afa}

\NormalTok{read.aln <-}\StringTok{ }\KeywordTok{read.alignment}\NormalTok{(}\DataTypeTok{file =} \StringTok{"./data/p.isolates.afa"}\NormalTok{, }\DataTypeTok{format =} \StringTok{"fasta"}\NormalTok{)}

\NormalTok{p.DNAbin <-}\StringTok{ }\KeywordTok{as.DNAbin}\NormalTok{(read.aln)}

\NormalTok{window <-}\StringTok{ }\NormalTok{p.DNAbin[, }\DecValTok{200}\OperatorTok{:}\DecValTok{500}\NormalTok{]}

\KeywordTok{image.DNAbin}\NormalTok{(window, }\DataTypeTok{cex.lab =} \FloatTok{0.50}\NormalTok{)}
\end{Highlighting}
\end{Shaded}

\includegraphics{11.PhyloTraits_Worksheet_files/figure-latex/unnamed-chunk-2-1.pdf}

\textbf{\emph{Question 2}}: Make some observations about the
\texttt{muscle} alignment of the 16S rRNA gene sequences for our
bacterial isolates and the outgroup, \emph{Methanosarcina}, a member of
the domain Archaea. Move along the alignment by changing the values in
the \texttt{window} object.

\begin{enumerate}
\def\labelenumi{\alph{enumi}.}
\tightlist
\item
  Approximately how long are our sequence reads?\\
\item
  What regions do you think would are appropriate for phylogenetic
  inference and why?
\end{enumerate}

\begin{quote}
\textbf{\emph{Answer 2a}}: I'm not quite sure how to tell\ldots{}
wouldn't it just be as long as whatever gene we are looking at? In that
case, I'd say our sequence reads are about 200 bp long, because that's
how long all the sequences align fairly well before becoming quite
different. \textbf{\emph{Answer 2b}}: I think the region between about
50 and 250 bp would be good for phylogenetic inference because the
sequences are fairly similar, but different enough that we would be able
to draw some relationships based on those differences.
\end{quote}

\subsection{4) MAKING A PHYLOGENETIC
TREE}\label{making-a-phylogenetic-tree}

Once you have aligned your sequences, the next step is to construct a
phylogenetic tree. Not only is a phylogenetic tree effective for
visualizing the evolutionary relationship among taxa, but as you will
see later, the information that goes into a phylogenetic tree is needed
for downstream analysis.

\subsubsection{A. Neighbor Joining
Trees}\label{a.-neighbor-joining-trees}

In the R code chunk below, do the following:\\
1. calculate the distance matrix using \texttt{model\ =\ "raw"},\\
2. create a Neighbor Joining tree based on these distances,\\
3. define ``Methanosarcina'' as the outgroup and root the tree, and\\
4. plot the rooted tree.

\begin{Shaded}
\begin{Highlighting}[]
\NormalTok{seq.dist.raw <-}\StringTok{ }\KeywordTok{dist.dna}\NormalTok{(p.DNAbin, }\DataTypeTok{model =} \StringTok{"raw"}\NormalTok{, }\DataTypeTok{pairwise.deletion =} \OtherTok{FALSE}\NormalTok{)}

\NormalTok{nj.tree <-}\StringTok{ }\KeywordTok{bionj}\NormalTok{(seq.dist.raw)}
\NormalTok{outgroup <-}\StringTok{ }\KeywordTok{match}\NormalTok{(}\StringTok{"Methanosarcina"}\NormalTok{, nj.tree}\OperatorTok{$}\NormalTok{tip.label)}
\NormalTok{nj.rooted <-}\StringTok{ }\KeywordTok{root}\NormalTok{(nj.tree, outgroup, }\DataTypeTok{resolve.root =} \OtherTok{TRUE}\NormalTok{)}
\KeywordTok{par}\NormalTok{(}\DataTypeTok{mar =} \KeywordTok{c}\NormalTok{(}\DecValTok{1}\NormalTok{,}\DecValTok{1}\NormalTok{,}\DecValTok{2}\NormalTok{,}\DecValTok{1}\NormalTok{) }\OperatorTok{+}\StringTok{ }\FloatTok{0.1}\NormalTok{)}
\KeywordTok{plot.phylo}\NormalTok{(nj.rooted, }\DataTypeTok{main =} \StringTok{"Neighbor Joining Tree"}\NormalTok{, }\StringTok{"phylogram"}\NormalTok{, }\DataTypeTok{use.edge.length =} \OtherTok{FALSE}\NormalTok{,}
           \DataTypeTok{direction =} \StringTok{"right"}\NormalTok{, }\DataTypeTok{cex =} \FloatTok{0.6}\NormalTok{, }\DataTypeTok{label.offset =} \DecValTok{1}\NormalTok{)}
\KeywordTok{add.scale.bar}\NormalTok{(}\DataTypeTok{cex =} \FloatTok{0.7}\NormalTok{)}
\KeywordTok{grid}\NormalTok{(}\KeywordTok{ncol}\NormalTok{(window), }\KeywordTok{nrow}\NormalTok{(window), }\DataTypeTok{col =} \StringTok{"lightgrey"}\NormalTok{)}
\end{Highlighting}
\end{Shaded}

\includegraphics{11.PhyloTraits_Worksheet_files/figure-latex/unnamed-chunk-3-1.pdf}

\textbf{\emph{Question 3}}: What are the advantages and disadvantages of
making a neighbor joining tree?

\begin{quote}
\textbf{\emph{Answer 3}}: Neighbor joining trees are convenient because
they are quick to make and give a good snapshot of the general
relationships among the taxa at hand. However, it generally can only be
used as a snapshot, only a starting point for establishing the most
accurate tree. The neighbor joining tree method does not take multiple
substitutions into account the same site. They also don't account for
substitution biases.
\end{quote}

\subsubsection{B) SUBSTITUTION MODELS OF DNA
EVOLUTION}\label{b-substitution-models-of-dna-evolution}

In the R code chunk below, do the following:\\
1. make a second distance matrix based on the Felsenstein 84
substitution model,\\
2. create a saturation plot to compare the \emph{raw} and
\emph{Felsenstein (F84)} substitution models,\\
3. make Neighbor Joining trees for both, and\\
4. create a cophylogenetic plot to compare the topologies of the trees.

\begin{Shaded}
\begin{Highlighting}[]
\NormalTok{seq.dist.F84 <-}\StringTok{ }\KeywordTok{dist.dna}\NormalTok{(p.DNAbin, }\DataTypeTok{model =} \StringTok{"F84"}\NormalTok{, }\DataTypeTok{pairwise.deletion =} \OtherTok{FALSE}\NormalTok{)}

\KeywordTok{par}\NormalTok{(}\DataTypeTok{mar =} \KeywordTok{c}\NormalTok{(}\DecValTok{5}\NormalTok{, }\DecValTok{5}\NormalTok{, }\DecValTok{2}\NormalTok{, }\DecValTok{1}\NormalTok{) }\OperatorTok{+}\StringTok{ }\FloatTok{0.1}\NormalTok{)}
\KeywordTok{plot}\NormalTok{(seq.dist.raw, seq.dist.F84,}
     \DataTypeTok{pch =} \DecValTok{20}\NormalTok{, }\DataTypeTok{col =} \StringTok{"red"}\NormalTok{, }\DataTypeTok{las =} \DecValTok{1}\NormalTok{, }\DataTypeTok{asp =} \DecValTok{1}\NormalTok{, }\DataTypeTok{xlim =} \KeywordTok{c}\NormalTok{(}\DecValTok{0}\NormalTok{, }\FloatTok{0.7}\NormalTok{), }\DataTypeTok{ylim =} \KeywordTok{c}\NormalTok{(}\DecValTok{0}\NormalTok{, }\FloatTok{0.7}\NormalTok{), }
     \DataTypeTok{xlab =} \StringTok{"Raw Distance"}\NormalTok{, }\DataTypeTok{ylab =} \StringTok{"F84 Distance"}\NormalTok{) }
\KeywordTok{abline}\NormalTok{(}\DataTypeTok{b =} \DecValTok{1}\NormalTok{, }\DataTypeTok{a =} \DecValTok{0}\NormalTok{, }\DataTypeTok{lty =} \DecValTok{2}\NormalTok{)}
\KeywordTok{text}\NormalTok{(}\FloatTok{0.65}\NormalTok{, }\FloatTok{0.6}\NormalTok{, }\StringTok{"1:1"}\NormalTok{)}
\end{Highlighting}
\end{Shaded}

\includegraphics{11.PhyloTraits_Worksheet_files/figure-latex/unnamed-chunk-4-1.pdf}

\begin{Shaded}
\begin{Highlighting}[]
\NormalTok{raw.tree <-}\KeywordTok{bionj}\NormalTok{(seq.dist.raw)}
\NormalTok{F84.tree <-}\StringTok{ }\KeywordTok{bionj}\NormalTok{(seq.dist.F84)}
\NormalTok{raw.outgroup <-}\StringTok{ }\KeywordTok{match}\NormalTok{(}\StringTok{"Methanosarcina"}\NormalTok{, raw.tree}\OperatorTok{$}\NormalTok{tip.label)}
\NormalTok{F84.outgroup <-}\StringTok{ }\KeywordTok{match}\NormalTok{(}\StringTok{"Methanosarcina"}\NormalTok{, F84.tree}\OperatorTok{$}\NormalTok{tip.label)}
\NormalTok{raw.rooted <-}\StringTok{ }\KeywordTok{root}\NormalTok{(raw.tree, raw.outgroup, }\DataTypeTok{resolve.root =} \OtherTok{TRUE}\NormalTok{)}
\NormalTok{F84.rooted <-}\StringTok{ }\KeywordTok{root}\NormalTok{(F84.tree, F84.outgroup, }\DataTypeTok{resolve.root =} \OtherTok{TRUE}\NormalTok{)}
\KeywordTok{layout}\NormalTok{(}\KeywordTok{matrix}\NormalTok{(}\KeywordTok{c}\NormalTok{(}\DecValTok{1}\NormalTok{,}\DecValTok{2}\NormalTok{), }\DecValTok{1}\NormalTok{, }\DecValTok{2}\NormalTok{), }\DataTypeTok{width =} \KeywordTok{c}\NormalTok{(}\DecValTok{1}\NormalTok{, }\DecValTok{1}\NormalTok{))}
\KeywordTok{par}\NormalTok{(}\DataTypeTok{mar =} \KeywordTok{c}\NormalTok{(}\DecValTok{1}\NormalTok{, }\DecValTok{1}\NormalTok{, }\DecValTok{2}\NormalTok{, }\DecValTok{0}\NormalTok{))}
\KeywordTok{plot.phylo}\NormalTok{(raw.rooted, }\DataTypeTok{type =} \StringTok{"phylogram"}\NormalTok{, }\DataTypeTok{direction =} \StringTok{"right"}\NormalTok{, }\DataTypeTok{show.tip.label =} \OtherTok{TRUE}\NormalTok{,}
           \DataTypeTok{use.edge.length =} \OtherTok{FALSE}\NormalTok{, }\DataTypeTok{adj =} \FloatTok{0.5}\NormalTok{, }\DataTypeTok{cex =} \FloatTok{0.6}\NormalTok{, }\DataTypeTok{label.offset =} \DecValTok{2}\NormalTok{, }\DataTypeTok{main =} \StringTok{"Raw"}\NormalTok{)}
\KeywordTok{par}\NormalTok{(}\DataTypeTok{mar =} \KeywordTok{c}\NormalTok{(}\DecValTok{1}\NormalTok{, }\DecValTok{0}\NormalTok{, }\DecValTok{2}\NormalTok{, }\DecValTok{1}\NormalTok{))}
\KeywordTok{plot.phylo}\NormalTok{(F84.rooted, }\DataTypeTok{type =} \StringTok{"phylogram"}\NormalTok{, }\DataTypeTok{direction =} \StringTok{"left"}\NormalTok{, }\DataTypeTok{show.tip.label =} \OtherTok{TRUE}\NormalTok{,}
           \DataTypeTok{use.edge.length =} \OtherTok{FALSE}\NormalTok{, }\DataTypeTok{adj =} \FloatTok{0.5}\NormalTok{, }\DataTypeTok{cex =} \FloatTok{0.6}\NormalTok{, }\DataTypeTok{label.offset =} \DecValTok{2}\NormalTok{, }\DataTypeTok{main =} \StringTok{"F84"}\NormalTok{)}
\end{Highlighting}
\end{Shaded}

\includegraphics{11.PhyloTraits_Worksheet_files/figure-latex/unnamed-chunk-4-2.pdf}

In the R code chunk below, do the following:\\
1. pick another substitution model,\\
2. create a distance matrix and tree for this model,\\
3. make a saturation plot that compares that model to the
\emph{Felsenstein (F84)} model,\\
4. make a cophylogenetic plot that compares the topologies of both
models, and\\
5. be sure to format, add appropriate labels, and customize each plot.

\begin{Shaded}
\begin{Highlighting}[]
\NormalTok{seq.dist.JC69 <-}\StringTok{ }\KeywordTok{dist.dna}\NormalTok{(p.DNAbin, }\DataTypeTok{model =} \StringTok{"JC69"}\NormalTok{, }\DataTypeTok{pairwise.deletion =} \OtherTok{FALSE}\NormalTok{)}

\KeywordTok{par}\NormalTok{(}\DataTypeTok{mar =} \KeywordTok{c}\NormalTok{(}\DecValTok{5}\NormalTok{, }\DecValTok{5}\NormalTok{, }\DecValTok{2}\NormalTok{, }\DecValTok{1}\NormalTok{) }\OperatorTok{+}\StringTok{ }\FloatTok{0.1}\NormalTok{)}
\KeywordTok{plot}\NormalTok{(seq.dist.raw, seq.dist.F84,}
     \DataTypeTok{pch =} \DecValTok{20}\NormalTok{, }\DataTypeTok{col =} \StringTok{"red"}\NormalTok{, }\DataTypeTok{las =} \DecValTok{1}\NormalTok{, }\DataTypeTok{asp =} \DecValTok{1}\NormalTok{, }\DataTypeTok{xlim =} \KeywordTok{c}\NormalTok{(}\DecValTok{0}\NormalTok{, }\FloatTok{0.7}\NormalTok{), }\DataTypeTok{ylim =} \KeywordTok{c}\NormalTok{(}\DecValTok{0}\NormalTok{, }\FloatTok{0.7}\NormalTok{), }
     \DataTypeTok{xlab =} \StringTok{"Raw Distance"}\NormalTok{, }\DataTypeTok{ylab =} \StringTok{"JC69 Distance"}\NormalTok{) }
\KeywordTok{abline}\NormalTok{(}\DataTypeTok{b =} \DecValTok{1}\NormalTok{, }\DataTypeTok{a =} \DecValTok{0}\NormalTok{, }\DataTypeTok{lty =} \DecValTok{2}\NormalTok{)}
\KeywordTok{text}\NormalTok{(}\FloatTok{0.65}\NormalTok{, }\FloatTok{0.6}\NormalTok{, }\StringTok{"1:1"}\NormalTok{)}
\end{Highlighting}
\end{Shaded}

\includegraphics{11.PhyloTraits_Worksheet_files/figure-latex/unnamed-chunk-5-1.pdf}

\begin{Shaded}
\begin{Highlighting}[]
\NormalTok{F84.tree <-}\StringTok{ }\KeywordTok{bionj}\NormalTok{(seq.dist.F84)}
\NormalTok{JC69.tree <-}\StringTok{ }\KeywordTok{bionj}\NormalTok{(seq.dist.JC69)}
\NormalTok{F84.outgroup <-}\StringTok{ }\KeywordTok{match}\NormalTok{(}\StringTok{"Methanosarcina"}\NormalTok{, F84.tree}\OperatorTok{$}\NormalTok{tip.label)}
\NormalTok{JC69.outgroup <-}\StringTok{ }\KeywordTok{match}\NormalTok{(}\StringTok{"Methanosarcina"}\NormalTok{, JC69.tree}\OperatorTok{$}\NormalTok{tip.label)}
\NormalTok{F84.rooted <-}\StringTok{ }\KeywordTok{root}\NormalTok{(F84.tree, F84.outgroup, }\DataTypeTok{resolve.root =} \OtherTok{TRUE}\NormalTok{)}
\NormalTok{JC69.rooted <-}\StringTok{ }\KeywordTok{root}\NormalTok{(JC69.tree, JC69.outgroup, }\DataTypeTok{resolve.root =} \OtherTok{TRUE}\NormalTok{)}
\KeywordTok{layout}\NormalTok{(}\KeywordTok{matrix}\NormalTok{(}\KeywordTok{c}\NormalTok{(}\DecValTok{1}\NormalTok{,}\DecValTok{2}\NormalTok{), }\DecValTok{1}\NormalTok{, }\DecValTok{2}\NormalTok{), }\DataTypeTok{width =} \KeywordTok{c}\NormalTok{(}\DecValTok{1}\NormalTok{, }\DecValTok{1}\NormalTok{))}
\KeywordTok{par}\NormalTok{(}\DataTypeTok{mar =} \KeywordTok{c}\NormalTok{(}\DecValTok{1}\NormalTok{, }\DecValTok{1}\NormalTok{, }\DecValTok{2}\NormalTok{, }\DecValTok{0}\NormalTok{))}
\KeywordTok{plot.phylo}\NormalTok{(F84.rooted, }\DataTypeTok{type =} \StringTok{"phylogram"}\NormalTok{, }\DataTypeTok{direction =} \StringTok{"right"}\NormalTok{, }\DataTypeTok{show.tip.label =} \OtherTok{TRUE}\NormalTok{,}
           \DataTypeTok{use.edge.length =} \OtherTok{FALSE}\NormalTok{, }\DataTypeTok{adj =} \FloatTok{0.5}\NormalTok{, }\DataTypeTok{cex =} \FloatTok{0.6}\NormalTok{, }\DataTypeTok{label.offset =} \DecValTok{2}\NormalTok{, }\DataTypeTok{main =} \StringTok{"F84"}\NormalTok{)}
\KeywordTok{par}\NormalTok{(}\DataTypeTok{mar =} \KeywordTok{c}\NormalTok{(}\DecValTok{1}\NormalTok{, }\DecValTok{0}\NormalTok{, }\DecValTok{2}\NormalTok{, }\DecValTok{1}\NormalTok{))}
\KeywordTok{plot.phylo}\NormalTok{(JC69.rooted, }\DataTypeTok{type =} \StringTok{"phylogram"}\NormalTok{, }\DataTypeTok{direction =} \StringTok{"left"}\NormalTok{, }\DataTypeTok{show.tip.label =} \OtherTok{TRUE}\NormalTok{,}
           \DataTypeTok{use.edge.length =} \OtherTok{FALSE}\NormalTok{, }\DataTypeTok{adj =} \FloatTok{0.5}\NormalTok{, }\DataTypeTok{cex =} \FloatTok{0.6}\NormalTok{, }\DataTypeTok{label.offset =} \DecValTok{2}\NormalTok{, }\DataTypeTok{main =} \StringTok{"JC69"}\NormalTok{)}
\end{Highlighting}
\end{Shaded}

\includegraphics{11.PhyloTraits_Worksheet_files/figure-latex/unnamed-chunk-5-2.pdf}

\textbf{\emph{Question 4}}:

\begin{enumerate}
\def\labelenumi{\alph{enumi}.}
\tightlist
\item
  Describe the substitution model that you chose. What assumptions does
  it make and how does it compare to the F84 model?
\item
  Using the saturation plot and cophylogenetic plots from above,
  describe how your choice of substitution model affects your
  phylogenetic reconstruction. If the plots are inconsistent with one
  another, explain why.
\item
  How does your model compare to the \emph{F84} model and what does this
  tell you about the substitution rates of nucleotide transitions?
\end{enumerate}

\begin{quote}
\textbf{\emph{Answer 4a}}: I decided to use the Jukes-Cantor model from
1969. It's a very simple model that assumes that all nucleotides are
equally frequent and that there's the same substitution probability for
all nucleotides. The F84 model takes the difference in transitions and
transversions into account.\\
\textbf{\emph{Answer 4b}}: The F84 plot differs slightly from the Raw
distances, but only slightly. This small inconsistency is likely due to
substitution bias taken into account by the F84 model.\\
\textbf{\emph{Answer 4c}}: Interestingly, the the Jukes-Cantor and
Felsenstein 84 models output identical trees. This tells me that the
substitution rates of nucleotide transitions are equally probable in
this particular data set.
\end{quote}

\subsubsection{C) ANALYZING A MAXIMUM LIKELIHOOD
TREE}\label{c-analyzing-a-maximum-likelihood-tree}

In the R code chunk below, do the following:\\
1. Read in the maximum likelihood phylogenetic tree used in the handout.
2. Plot bootstrap support values onto the tree

\begin{Shaded}
\begin{Highlighting}[]
\CommentTok{#Every time I tried to run this section, it would stop on line 226 and freeze up. }
\CommentTok{#I let it run for a couple hours and it still just got stuck every time. }

\CommentTok{#dist.topo(raw.rooted, F84.rooted, method = "score")}

\CommentTok{#p.DNAbin.phyDat <- read.phyDat("./data/p.isolates.afa", format = "fasta", type = "DNA")}
\CommentTok{#fit <- pml(nj.rooted, data = p.DNAbin.phyDat)}
\CommentTok{#fitJC <- optim.pml(fit, TRUE)}
\CommentTok{#fitGTR <- optim.pml(fitGTR, model = "GTR", optInv = TRUE, optGamma = TRUE)}
\CommentTok{#anova(fitJC, fitGTR)}
\CommentTok{#AIC(fitJC)}
\CommentTok{#AIC(fitGTR)}
\CommentTok{#bs = bootstrap.pml(fitJC, bs = 100, optNni = TRUE,}
 \CommentTok{#                  control = pmlcontrol(trace = 0))}
\CommentTok{#ml.bootstrap <- read.tree("./data/ml_tree/RAxML_bipartitions.T1")}
\CommentTok{#par(mar = c(1,1,2,1) + 0.1)}
\CommentTok{#plot.phylo(ml.bootstrap, type = "phylogram", irection = "right", show.tip.label = TRUE,}
\CommentTok{#           use.edge.length = FALSE, cex = 0.6, label.offset = 1, main = "Maximum Likelihood with Support") }
\CommentTok{#add.scale.bar(cex = 0.7)}
\CommentTok{#nodelabels(ml.bootstrap$node.label, font = 2, bg = "white", frame = "r", cex = 0.5)}
\end{Highlighting}
\end{Shaded}

\textbf{\emph{Question 5}}:

\begin{enumerate}
\def\labelenumi{\alph{enumi})}
\item
  How does the maximum likelihood tree compare the to the
  neighbor-joining tree in the handout? If the plots seem to be
  inconsistent with one another, explain what gives rise to the
  differences.
\item
  Why do we bootstrap our tree?
\item
  What do the bootstrap values tell you?
\item
  Which branches have very low support?
\item
  Should we trust these branches?
\end{enumerate}

\begin{quote}
\textbf{\emph{Answer 5a}}: The groupings within the neighbor joining
tree and maximum likelihood tree are similar, but not identical. This
would essentially come down to the processes used to generate the trees.
The tree with the greatest parsimony may not be the tree that arises
from comparing taxa one by one as is done in creating a neighbor joining
tree. \textbf{\emph{Answer 5b}}: We bootstrap our tree in order to
establish its reliability, but ensuring we get the same thing every time
we run the statistical analysis. \textbf{\emph{Answer 5c}}: The
bootstrap values tell us how often the branch is in the same place as we
re-sample our data. Since it is a percentage, it will be a value between
0 and 100. \textbf{\emph{Answer 5d}}: We get pretty low support in the
cluster of taxa including WG42, LL43F, etc., as low as 21-22\%.\\
\textbf{\emph{Answer 5e}}: We should probably not trust these branches,
there is often variation in how those branches and taxa relate to one
another in the tree.
\end{quote}

\subsection{5) INTEGRATING TRAITS AND
PHYLOGENY}\label{integrating-traits-and-phylogeny}

\subsubsection{A. Loading Trait
Database}\label{a.-loading-trait-database}

In the R code chunk below, do the following:\\
1. import the raw phosphorus growth data, and\\
2. standardize the data for each strain by the sum of growth rates.

\begin{Shaded}
\begin{Highlighting}[]
\NormalTok{p.growth <-}\StringTok{ }\KeywordTok{read.table}\NormalTok{(}\StringTok{"./data/p.isolates.raw.growth.txt"}\NormalTok{, }\DataTypeTok{sep =} \StringTok{"}\CharTok{\textbackslash{}t}\StringTok{"}\NormalTok{, }\DataTypeTok{header =} \OtherTok{TRUE}\NormalTok{,}
                       \DataTypeTok{row.names =} \DecValTok{1}\NormalTok{)}
\NormalTok{p.growth.std <-}\StringTok{ }\NormalTok{p.growth }\OperatorTok{/}\StringTok{ }\NormalTok{(}\KeywordTok{apply}\NormalTok{(p.growth, }\DecValTok{1}\NormalTok{, sum))}
\end{Highlighting}
\end{Shaded}

\subsubsection{B. Trait Manipulations}\label{b.-trait-manipulations}

In the R code chunk below, do the following:\\
1. calculate the maximum growth rate (\(\mu_{max}\)) of each isolate
across all phosphorus types,\\
2. create a function that calculates niche breadth (\emph{nb}), and\\
3. use this function to calculate \emph{nb} for each isolate.

\begin{Shaded}
\begin{Highlighting}[]
\NormalTok{umax <-}\StringTok{ }\NormalTok{(}\KeywordTok{apply}\NormalTok{(p.growth, }\DecValTok{1}\NormalTok{, max))}
\NormalTok{levins <-}\StringTok{ }\ControlFlowTok{function}\NormalTok{(}\DataTypeTok{p_xi =} \StringTok{""}\NormalTok{)\{}
\NormalTok{  p =}\StringTok{ }\DecValTok{0}
  \ControlFlowTok{for}\NormalTok{ (i }\ControlFlowTok{in}\NormalTok{ p_xi)\{}
\NormalTok{    p =}\StringTok{ }\NormalTok{p }\OperatorTok{+}\StringTok{ }\NormalTok{i}\OperatorTok{^}\DecValTok{2}
\NormalTok{  \}}
\NormalTok{  nb =}\StringTok{ }\DecValTok{1} \OperatorTok{/}\StringTok{ }\NormalTok{(}\KeywordTok{length}\NormalTok{(p_xi) }\OperatorTok{*}\StringTok{ }\NormalTok{p)}
  \KeywordTok{return}\NormalTok{(nb)}
\NormalTok{\}}
\NormalTok{nb <-}\StringTok{ }\KeywordTok{as.matrix}\NormalTok{(}\KeywordTok{levins}\NormalTok{(p.growth.std))}
\KeywordTok{rownames}\NormalTok{(nb) <-}\StringTok{ }\KeywordTok{row.names}\NormalTok{(p.growth)}
\KeywordTok{colnames}\NormalTok{(nb) <-}\StringTok{ }\KeywordTok{c}\NormalTok{(}\StringTok{"NB"}\NormalTok{)}
\end{Highlighting}
\end{Shaded}

\subsubsection{C. Visualizing Traits on
Trees}\label{c.-visualizing-traits-on-trees}

In the R code chunk below, do the following:\\
1. pick your favorite substitution model and make a Neighbor Joining
tree,\\
2. define your outgroup and root the tree, and\\
3. remove the outgroup branch.

\begin{Shaded}
\begin{Highlighting}[]
\NormalTok{nj.tree <-}\StringTok{ }\KeywordTok{bionj}\NormalTok{(seq.dist.F84)}
\NormalTok{nj.outgroup <-}\StringTok{ }\KeywordTok{match}\NormalTok{(}\StringTok{"Methanosarcina"}\NormalTok{, nj.tree}\OperatorTok{$}\NormalTok{tip.label)}
\NormalTok{nj.rooted <-}\StringTok{ }\KeywordTok{root}\NormalTok{(nj.tree, nj.outgroup, }\DataTypeTok{resolve.root =} \OtherTok{TRUE}\NormalTok{)}
\NormalTok{nj.rooted <-}\StringTok{ }\KeywordTok{drop.tip}\NormalTok{(nj.rooted, }\StringTok{"Methanosarcina"}\NormalTok{)}
\end{Highlighting}
\end{Shaded}

In the R code chunk below, do the following:\\
1. define a color palette (use something other than ``YlOrRd''),\\
2. map the phosphorus traits onto your phylogeny,\\
3. map the \emph{nb} trait on to your phylogeny, and\\
4. customize the plots as desired (use \texttt{help(table.phylo4d)} to
learn about the options).

\begin{Shaded}
\begin{Highlighting}[]
\NormalTok{mypalette <-}\StringTok{ }\KeywordTok{colorRampPalette}\NormalTok{(}\KeywordTok{brewer.pal}\NormalTok{(}\DecValTok{9}\NormalTok{, }\StringTok{"BuPu"}\NormalTok{))}

\KeywordTok{par}\NormalTok{(}\DataTypeTok{mar =} \KeywordTok{c}\NormalTok{(}\DecValTok{1}\NormalTok{,}\DecValTok{1}\NormalTok{,}\DecValTok{1}\NormalTok{,}\DecValTok{1}\NormalTok{) }\OperatorTok{+}\StringTok{ }\FloatTok{0.1}\NormalTok{)}
\NormalTok{x <-}\StringTok{ }\KeywordTok{phylo4d}\NormalTok{(nj.rooted, p.growth.std)}
\KeywordTok{table.phylo4d}\NormalTok{(x, }\DataTypeTok{treetype =} \StringTok{"phylo"}\NormalTok{, }\DataTypeTok{symbol =} \StringTok{"colors"}\NormalTok{, }\DataTypeTok{show.node =} \OtherTok{TRUE}\NormalTok{,}
              \DataTypeTok{cex.label =} \FloatTok{0.5}\NormalTok{, }\DataTypeTok{scale =} \OtherTok{FALSE}\NormalTok{, }\DataTypeTok{use.edge.length =} \OtherTok{FALSE}\NormalTok{,}
              \DataTypeTok{edge.color =} \StringTok{"black"}\NormalTok{, }\DataTypeTok{edge.width =} \DecValTok{2}\NormalTok{, }\DataTypeTok{box =} \OtherTok{FALSE}\NormalTok{,}
              \DataTypeTok{col =} \KeywordTok{mypalette}\NormalTok{(}\DecValTok{25}\NormalTok{), }\DataTypeTok{pch =} \DecValTok{15}\NormalTok{, }\DataTypeTok{cex.symbol =} \FloatTok{1.25}\NormalTok{,}
              \DataTypeTok{ratio.tree =} \FloatTok{0.5}\NormalTok{, }\DataTypeTok{cex.legend =} \FloatTok{1.5}\NormalTok{, }\DataTypeTok{center =} \OtherTok{FALSE}\NormalTok{)}
\end{Highlighting}
\end{Shaded}

\includegraphics{11.PhyloTraits_Worksheet_files/figure-latex/unnamed-chunk-10-1.pdf}

\begin{Shaded}
\begin{Highlighting}[]
\KeywordTok{par}\NormalTok{(}\DataTypeTok{mar =} \KeywordTok{c}\NormalTok{(}\DecValTok{1}\NormalTok{,}\DecValTok{5}\NormalTok{,}\DecValTok{1}\NormalTok{,}\DecValTok{5}\NormalTok{) }\OperatorTok{+}\StringTok{ }\FloatTok{0.1}\NormalTok{)}
\NormalTok{x.nb <-}\StringTok{ }\KeywordTok{phylo4d}\NormalTok{(nj.rooted, nb)}
\KeywordTok{table.phylo4d}\NormalTok{(x.nb, }\DataTypeTok{treetype =} \StringTok{"phylo"}\NormalTok{, }\DataTypeTok{symbol =} \StringTok{"colors"}\NormalTok{, }\DataTypeTok{show.node =} \OtherTok{TRUE}\NormalTok{,}
              \DataTypeTok{cex.label =} \FloatTok{0.5}\NormalTok{, }\DataTypeTok{scale =} \OtherTok{FALSE}\NormalTok{, }\DataTypeTok{use.edge.length =} \OtherTok{FALSE}\NormalTok{,}
              \DataTypeTok{edge.color =} \StringTok{"black"}\NormalTok{, }\DataTypeTok{edge.width =} \DecValTok{2}\NormalTok{, }\DataTypeTok{box =} \OtherTok{FALSE}\NormalTok{,}
              \DataTypeTok{col=}\KeywordTok{mypalette}\NormalTok{(}\DecValTok{25}\NormalTok{), }\DataTypeTok{pch =} \DecValTok{15}\NormalTok{, }\DataTypeTok{cex.symbol =} \FloatTok{1.25}\NormalTok{, }\DataTypeTok{var.label =}\NormalTok{ (}\StringTok{"NB"}\NormalTok{), }
              \DataTypeTok{ratio.tree =} \FloatTok{0.90}\NormalTok{, }\DataTypeTok{cex.legend =} \FloatTok{1.5}\NormalTok{, }\DataTypeTok{center =} \OtherTok{FALSE}\NormalTok{)}
\end{Highlighting}
\end{Shaded}

\includegraphics{11.PhyloTraits_Worksheet_files/figure-latex/unnamed-chunk-10-2.pdf}

\textbf{\emph{Question 6}}:

\begin{enumerate}
\def\labelenumi{\alph{enumi})}
\item
  Make a hypothesis that would support a generalist-specialist
  trade-off.
\item
  What kind of patterns would you expect to see from growth rate and
  niche breadth values that would support this hypothesis?
\end{enumerate}

\begin{quote}
\textbf{\emph{Answer 6a}}: I would hypothesize that generalist species
would have a high niche breadth but a low growth rate, as efficiency in
using various phosphorous-based resources would likely be decreased if
you are trying to use so many different compounds. Specialist species
would likely have low niche breadth, but a high growth rate in the
niches they are able to fill because they can more efficiently take
advantage of the phosphorous-based resource they specialize in breaking
down.\\
\textbf{\emph{Answer 6b}}: As I sort of explained above, we would expect
to see (and it appears we do see this), some species that have a
moderate growth rate across all phosphorous resource, and others that
have a higher growth rate across one or a few phosphorous resources.
(Phosphorous resources in this case being what defines the niche.)
\end{quote}

\subsection{6) HYPOTHESIS TESTING}\label{hypothesis-testing}

\subsubsection{A) Phylogenetic Signal: Pagel's
Lambda}\label{a-phylogenetic-signal-pagels-lambda}

In the R code chunk below, do the following:\\
1. create two rescaled phylogenetic trees using lambda values of 0.5 and
0,\\
2. plot your original tree and the two scaled trees, and\\
3. label and customize the trees as desired.

\begin{Shaded}
\begin{Highlighting}[]
\NormalTok{nj.lambda.}\DecValTok{5}\NormalTok{ <-}\StringTok{ }\KeywordTok{rescale}\NormalTok{(nj.rooted, }\StringTok{"lambda"}\NormalTok{, }\FloatTok{0.5}\NormalTok{)}
\NormalTok{nj.lambda.}\DecValTok{0}\NormalTok{ <-}\StringTok{ }\KeywordTok{rescale}\NormalTok{(nj.rooted, }\StringTok{"lambda"}\NormalTok{, }\DecValTok{0}\NormalTok{)}
\KeywordTok{layout}\NormalTok{(}\KeywordTok{matrix}\NormalTok{(}\KeywordTok{c}\NormalTok{(}\DecValTok{1}\NormalTok{,}\DecValTok{2}\NormalTok{,}\DecValTok{3}\NormalTok{), }\DecValTok{1}\NormalTok{, }\DecValTok{3}\NormalTok{), }\DataTypeTok{width =} \KeywordTok{c}\NormalTok{(}\DecValTok{1}\NormalTok{, }\DecValTok{1}\NormalTok{, }\DecValTok{1}\NormalTok{))}
\KeywordTok{par}\NormalTok{(}\DataTypeTok{mar =} \KeywordTok{c}\NormalTok{(}\DecValTok{1}\NormalTok{,}\FloatTok{0.5}\NormalTok{,}\DecValTok{2}\NormalTok{,}\FloatTok{0.5}\NormalTok{) }\OperatorTok{+}\StringTok{ }\FloatTok{0.1}\NormalTok{)}
\KeywordTok{plot}\NormalTok{(nj.rooted, }\DataTypeTok{main =} \StringTok{"Original tree, lambda = 1"}\NormalTok{, }\DataTypeTok{cex =} \FloatTok{0.7}\NormalTok{, }\DataTypeTok{adj =} \FloatTok{0.5}\NormalTok{)}
\KeywordTok{plot}\NormalTok{(nj.lambda.}\DecValTok{5}\NormalTok{, }\DataTypeTok{main =} \StringTok{"Rescaled tree, lambda = 0.5"}\NormalTok{, }\DataTypeTok{cex =} \FloatTok{0.7}\NormalTok{, }\DataTypeTok{adj =} \FloatTok{0.5}\NormalTok{)}
\KeywordTok{plot}\NormalTok{(nj.lambda.}\DecValTok{0}\NormalTok{, }\DataTypeTok{main =} \StringTok{"Rescaled tree, lambda = 0"}\NormalTok{, }\DataTypeTok{cex =} \FloatTok{0.7}\NormalTok{, }\DataTypeTok{adj =} \FloatTok{0.5}\NormalTok{)}
\end{Highlighting}
\end{Shaded}

\includegraphics{11.PhyloTraits_Worksheet_files/figure-latex/unnamed-chunk-11-1.pdf}

In the R code chunk below, do the following:\\
1. use the \texttt{fitContinuous()} function to compare your original
tree to the transformed trees.

\begin{Shaded}
\begin{Highlighting}[]
\KeywordTok{fitContinuous}\NormalTok{(nj.rooted, nb, }\DataTypeTok{model =} \StringTok{"lambda"}\NormalTok{)}
\end{Highlighting}
\end{Shaded}

\begin{verbatim}
## GEIGER-fitted comparative model of continuous data
##  fitted 'lambda' model parameters:
##  lambda = 0.020849
##  sigsq = 0.106492
##  z0 = 0.661368
## 
##  model summary:
##  log-likelihood = 21.661104
##  AIC = -37.322208
##  AICc = -36.636494
##  free parameters = 3
## 
## Convergence diagnostics:
##  optimization iterations = 100
##  failed iterations = 52
##  frequency of best fit = NA
## 
##  object summary:
##  'lik' -- likelihood function
##  'bnd' -- bounds for likelihood search
##  'res' -- optimization iteration summary
##  'opt' -- maximum likelihood parameter estimates
\end{verbatim}

\begin{Shaded}
\begin{Highlighting}[]
\KeywordTok{fitContinuous}\NormalTok{(nj.lambda.}\DecValTok{0}\NormalTok{, nb, }\DataTypeTok{model =} \StringTok{"lambda"}\NormalTok{)}
\end{Highlighting}
\end{Shaded}

\begin{verbatim}
## GEIGER-fitted comparative model of continuous data
##  fitted 'lambda' model parameters:
##  lambda = 0.000000
##  sigsq = 0.106395
##  z0 = 0.657777
## 
##  model summary:
##  log-likelihood = 21.652293
##  AIC = -37.304587
##  AICc = -36.618872
##  free parameters = 3
## 
## Convergence diagnostics:
##  optimization iterations = 100
##  failed iterations = 0
##  frequency of best fit = 0.83
## 
##  object summary:
##  'lik' -- likelihood function
##  'bnd' -- bounds for likelihood search
##  'res' -- optimization iteration summary
##  'opt' -- maximum likelihood parameter estimates
\end{verbatim}

\textbf{\emph{Question 7}}: There are two important outputs from the
\texttt{fitContinuous()} function that can help you interpret the
phylogenetic signal in trait data sets. a. Compare the lambda values of
the untransformed tree to the transformed (lambda = 0). b. Compare the
Akaike information criterion (AIC) scores of the two models. Which model
would you choose based off of AIC score (remember the criteria that the
difference in AIC values has to be at least 2)? c. Does this result
suggest that there's phylogenetic signal?

\begin{quote}
\textbf{\emph{Answer 7a}}: The lambda value of the original tree is
.021, and the lambda value of the transformed tree is 0.\\
\textbf{\emph{Answer 7b}}: The AIC value for the original tree is around
-37, and the lambda value of the transformed tree is the same. These
values mean the models are essentially identically, so it wouldn't
matter which one I choose. I suppose I would choose the original because
I can actually use it to draw relationships between different species.
\textbf{\emph{Answer 7c}}: This results suggests little to no
phylogenetic signal.
\end{quote}

\subsubsection{B) Phylogenetic Signal: Blomberg's
K}\label{b-phylogenetic-signal-blombergs-k}

In the R code chunk below, do the following:\\
1. correct tree branch-lengths to fix any zeros,\\
2. calculate Blomberg's K for each phosphorus resource using the
\texttt{phylosignal()} function,\\
3. use the Benjamini-Hochberg method to correct for false discovery
rate, and\\
4. calculate Blomberg's K for niche breadth using the
\texttt{phylosignal()} function.

\begin{Shaded}
\begin{Highlighting}[]
\NormalTok{nj.rooted}\OperatorTok{$}\NormalTok{edge.length <-}\StringTok{ }\NormalTok{nj.rooted}\OperatorTok{$}\NormalTok{edge.length }\OperatorTok{+}\StringTok{ }\DecValTok{10}\OperatorTok{^-}\DecValTok{7}
\NormalTok{p.phylosignal <-}\StringTok{ }\KeywordTok{matrix}\NormalTok{(}\OtherTok{NA}\NormalTok{, }\DecValTok{6}\NormalTok{, }\DecValTok{18}\NormalTok{)}
\KeywordTok{colnames}\NormalTok{(p.phylosignal) <-}\StringTok{ }\KeywordTok{colnames}\NormalTok{(p.growth.std)}
\KeywordTok{rownames}\NormalTok{(p.phylosignal) <-}\StringTok{ }\KeywordTok{c}\NormalTok{(}\StringTok{"K"}\NormalTok{, }\StringTok{"PIC.var.obs"}\NormalTok{, }\StringTok{"PIC.var.mean"}\NormalTok{,}
                             \StringTok{"PIC.var.P"}\NormalTok{, }\StringTok{"PIC.var.z"}\NormalTok{, }\StringTok{"PIC.P.BH"}\NormalTok{)}
\ControlFlowTok{for}\NormalTok{ (i }\ControlFlowTok{in} \DecValTok{1}\OperatorTok{:}\DecValTok{18}\NormalTok{)\{}
\NormalTok{  x <-}\StringTok{ }\KeywordTok{as.matrix}\NormalTok{(p.growth.std[ ,i, }\DataTypeTok{drop =} \OtherTok{FALSE}\NormalTok{])}
\NormalTok{  out <-}\StringTok{ }\KeywordTok{phylosignal}\NormalTok{(x, nj.rooted)}
\NormalTok{  p.phylosignal[}\DecValTok{1}\OperatorTok{:}\DecValTok{5}\NormalTok{, i] <-}\StringTok{ }\KeywordTok{round}\NormalTok{(}\KeywordTok{t}\NormalTok{(out), }\DecValTok{3}\NormalTok{)}
\NormalTok{\}}
\NormalTok{p.phylosignal[}\DecValTok{6}\NormalTok{, ] <-}\StringTok{ }\KeywordTok{round}\NormalTok{(}\KeywordTok{p.adjust}\NormalTok{(p.phylosignal[}\DecValTok{4}\NormalTok{, ], }\DataTypeTok{method =} \StringTok{"BH"}\NormalTok{), }\DecValTok{3}\NormalTok{)}
\NormalTok{signal.nb <-}\StringTok{ }\KeywordTok{phylosignal}\NormalTok{(nb, nj.rooted)}
\NormalTok{signal.nb}
\end{Highlighting}
\end{Shaded}

\begin{verbatim}
##              K PIC.variance.obs PIC.variance.rnd.mean PIC.variance.P
## 1 3.427719e-06         49966.78              50371.04          0.521
##   PIC.variance.Z
## 1    -0.02035049
\end{verbatim}

\textbf{\emph{Question 8}}: Using the K-values and associated p-values
(i.e., ``PIC.var.P''``) from the \texttt{phylosignal} output, answer the
following questions:

\begin{enumerate}
\def\labelenumi{\alph{enumi}.}
\tightlist
\item
  Is there significant phylogenetic signal for niche breadth or
  standardized growth on any of the phosphorus resources?\\
\item
  If there is significant phylogenetic signal, are the results
  suggestive of clustering or overdispersion?
\end{enumerate}

\begin{quote}
\textbf{\emph{Answer 8a}}: The phylogenetic signal is very low here, and
not significant with a p-value of 0.5.\\
\textbf{\emph{Answer 8b}}: There is no significant phylogenetic signal
here.
\end{quote}

\subsubsection{C. Calculate Dispersion of a
Trait}\label{c.-calculate-dispersion-of-a-trait}

In the R code chunk below, do the following:\\
1. turn the continuous growth data into categorical data,\\
2. add a column to the data with the isolate name,\\
3. combine the tree and trait data using the \texttt{comparative.data()}
function in \texttt{caper}, and\\
4. use \texttt{phylo.d()} to calculate \emph{D} on at least three
phosphorus traits.

\begin{Shaded}
\begin{Highlighting}[]
\NormalTok{p.growth.pa <-}\StringTok{ }\KeywordTok{as.data.frame}\NormalTok{((p.growth }\OperatorTok{>}\StringTok{ }\FloatTok{0.01}\NormalTok{) }\OperatorTok{*}\StringTok{ }\DecValTok{1}\NormalTok{)}
\KeywordTok{apply}\NormalTok{(p.growth.pa, }\DecValTok{2}\NormalTok{, sum)}
\end{Highlighting}
\end{Shaded}

\begin{verbatim}
##      AEP      PEP      G1P      G6P   MethCP      BGP      DNA     Peth 
##       20       38       35       34        3       35       19       21 
##    Pchol       B1     Phyt      SRP     cAMP      ATP PhenylCP    PolyP 
##       18       38       36       39       29       38        6       39 
##      GDP      GTP 
##       37       38
\end{verbatim}

\begin{Shaded}
\begin{Highlighting}[]
\NormalTok{p.growth.pa}\OperatorTok{$}\NormalTok{name <-}\StringTok{ }\KeywordTok{rownames}\NormalTok{(p.growth.pa)}

\NormalTok{p.traits <-}\StringTok{ }\KeywordTok{comparative.data}\NormalTok{(nj.rooted, p.growth.pa, }\StringTok{"name"}\NormalTok{)}
\KeywordTok{phylo.d}\NormalTok{(p.traits, }\DataTypeTok{binvar =}\NormalTok{ AEP)}
\end{Highlighting}
\end{Shaded}

\begin{verbatim}
## 
## Calculation of D statistic for the phylogenetic structure of a binary variable
## 
##   Data :  p.growth.pa
##   Binary variable :  AEP
##   Counts of states:  0 = 19
##                      1 = 20
##   Phylogeny :  nj.rooted
##   Number of permutations :  1000
## 
## Estimated D :  0.4687249
## Probability of E(D) resulting from no (random) phylogenetic structure :  0.005
## Probability of E(D) resulting from Brownian phylogenetic structure    :  0.025
\end{verbatim}

\begin{Shaded}
\begin{Highlighting}[]
\KeywordTok{phylo.d}\NormalTok{(p.traits, }\DataTypeTok{binvar =}\NormalTok{ PhenylCP)}
\end{Highlighting}
\end{Shaded}

\begin{verbatim}
## 
## Calculation of D statistic for the phylogenetic structure of a binary variable
## 
##   Data :  p.growth.pa
##   Binary variable :  PhenylCP
##   Counts of states:  0 = 33
##                      1 = 6
##   Phylogeny :  nj.rooted
##   Number of permutations :  1000
## 
## Estimated D :  0.8717874
## Probability of E(D) resulting from no (random) phylogenetic structure :  0.268
## Probability of E(D) resulting from Brownian phylogenetic structure    :  0.01
\end{verbatim}

\begin{Shaded}
\begin{Highlighting}[]
\KeywordTok{phylo.d}\NormalTok{(p.traits, }\DataTypeTok{binvar =}\NormalTok{ DNA)}
\end{Highlighting}
\end{Shaded}

\begin{verbatim}
## 
## Calculation of D statistic for the phylogenetic structure of a binary variable
## 
##   Data :  p.growth.pa
##   Binary variable :  DNA
##   Counts of states:  0 = 20
##                      1 = 19
##   Phylogeny :  nj.rooted
##   Number of permutations :  1000
## 
## Estimated D :  0.604519
## Probability of E(D) resulting from no (random) phylogenetic structure :  0.033
## Probability of E(D) resulting from Brownian phylogenetic structure    :  0.002
\end{verbatim}

\begin{Shaded}
\begin{Highlighting}[]
\KeywordTok{phylo.d}\NormalTok{(p.traits, }\DataTypeTok{binvar =}\NormalTok{ cAMP)}
\end{Highlighting}
\end{Shaded}

\begin{verbatim}
## 
## Calculation of D statistic for the phylogenetic structure of a binary variable
## 
##   Data :  p.growth.pa
##   Binary variable :  cAMP
##   Counts of states:  0 = 10
##                      1 = 29
##   Phylogeny :  nj.rooted
##   Number of permutations :  1000
## 
## Estimated D :  0.1354057
## Probability of E(D) resulting from no (random) phylogenetic structure :  0
## Probability of E(D) resulting from Brownian phylogenetic structure    :  0.341
\end{verbatim}

\textbf{\emph{Question 9}}: Using the estimates for \emph{D} and the
probabilities of each phylogenetic model, answer the following
questions:

\begin{enumerate}
\def\labelenumi{\alph{enumi}.}
\tightlist
\item
  Choose three phosphorus growth traits and test whether they are
  significantly clustered or overdispersed?\\
\item
  How do these results compare the results from the Blomberg's K
  analysis?\\
\item
  Discuss what factors might give rise to differences between the
  metrics.
\end{enumerate}

\begin{quote}
\textbf{\emph{Answer 9a}}: For AEP, D = 0.47. For PhenylCP, D = 0.86.
For cAMP, D = 0.14. All of these traits are overdispersed, with cAMP
more randomly clumped, AEP less randomly clumped, and PhenylCP the least
randomly clumped.\\
\textbf{\emph{Answer 9b}}: These results show more phylogenetic signal
than Blomberg's K did, more significance in the dispersal patterns.
\textbf{\emph{Answer 9c}}: From my Googling, it seems that the consensus
is that Blomberg's K is least accurate in testing for phylogenetic
signal, but it is good at looking at changes over time. In this case
we're looking at a static dataset, so it wouldn't be appropriate to use
Blomberg's K here.
\end{quote}

\subsection{7) PHYLOGENETIC REGRESSION}\label{phylogenetic-regression}

In the R code chunk below, do the following:\\
1. Load and clean the mammal phylogeny and trait dataset, 2. Fit a
linear model to the trait dataset, examining the relationship between
mass and BMR, 2. Fit a phylogenetic regression to the trait dataset,
taking into account the mammal supertree

\begin{Shaded}
\begin{Highlighting}[]
\CommentTok{#I had issues with this code chunk not recognizing the columns selected. I double and triple checked that}
\CommentTok{#the line and columns were typed correctly and matched the data. }

\CommentTok{#mammal.Tree <- read.tree("./data/mammal_best_super_tree_fritz2009.tre")}
\CommentTok{#mammal.data <- read.table("./data/mammal_BMR.txt", sep = "\textbackslash{}t", header = TRUE)}
\CommentTok{#mammal.data <- mammal.data[,c("Species", "BMR_.m102.hour.", "Body_mass_for_BMR_.gr.")]}
\CommentTok{#mammal.species <- array(mammal.data$Species)}
\CommentTok{#pruned.mammal.tree <- drop.tip(mammal.Tree, mammal.Tree$tip.label[-na.omit(math(mammal.species, mammal))])}
\CommentTok{#pruned.mammal.data <- mammal.data[mammal.data$Species %in% pruned.mammal.tree$tip.label,]}
\CommentTok{#rownames(pruned.mammal.data) <- pruned.mammal.data$Species}

\CommentTok{#fit <- lm(log10(BMR_.m102.hour.) ~ log10(Body_mass_for_BMR_.gr.), data = pruned.mammal.data)}
\CommentTok{#plot(log10(pruned.mammal.data$Body_mass_for_BMR_.gr.), log10(pruned.mammal.data$BMR_.m102.hour.), la)}
\CommentTok{#abline(a = fit$coefficients[1], b = fit4coefficients[2])}
\CommentTok{#b1 <- round(fit$coefficients[2],3)}
\CommentTok{#eqn <- bquote(italic(z) == .(b1))}
\CommentTok{#text(0.5, 4.5, eqn, pos = 4)}

\CommentTok{#fit.phy <- phylolm(log10(MR_.m102.hour.) ~ log10(Body_mass_for_BMR_.gr.), data = pruned.mammal.data,}
 \CommentTok{#                  pruned.mammmal.tree, model = 'lambda', boot = 0)}
\CommentTok{#plot(log10(pruned.mammal.data$Body_mass_for_BMR_.gr.), log10(pruned.mammal.data$BMR_.m102.hour.), las)}
\CommentTok{#abline(a = fit.phy$coefficients[1], b = fit.phy$coefficients[2])}
\CommentTok{#b1.phy <- round(fit.phy$coefficients[2],3)}
\CommentTok{#eqn <-bquote(italic(z) == .(b1.phy))}
\CommentTok{#text(0.5, 4.5, eqn, pos = 4)}
\end{Highlighting}
\end{Shaded}

\begin{enumerate}
\def\labelenumi{\alph{enumi}.}
\tightlist
\item
  Why do we need to correct for shared evolutionary history?
\item
  How does a phylogenetic regression differ from a standard linear
  regression?
\item
  Interpret the slope and fit of each model. Did accounting for shared
  evolutionary history improve or worsen the fit?
\item
  Try to come up with a scenario where the relationship between two
  variables would completely disappear when the underlying phylogeny is
  accounted for.
\end{enumerate}

\begin{quote}
\textbf{\emph{Answer 10a}}: Since we are looking at the relationship
between two variables, we need to correct for shared evolutionary
history to keep the variables independent from one another.\\
\textbf{\emph{Answer 10b}}: Phylogenetic regression is different from
linear regression because it has to take into account the branch lengths
of the phylogeny it's running the regression on. It also takes
phylogenetic signal into account, which a linear regression does not
need to do.\\
\textbf{\emph{Answer 10c}}: Accounting for shared evolutionary history
showed a stronger relationship between BMR and body mass, and improved
the fit with the data. \textbf{\emph{Answer 10d}}: Coming up with an
example is way harder than I thought it would be. :)
\end{quote}

\subsection{7) SYNTHESIS}\label{synthesis}

Work with members of your Team Project to obtain reference sequences for
10 or more taxa in your study. Sequences for plants, animals, and
microbes can found in a number of public repositories, but perhaps the
most commonly visited site is the National Center for Biotechnology
Information (NCBI) \url{https://www.ncbi.nlm.nih.gov/}. In almost all
cases, researchers must deposit their sequences in places like NCBI
before a paper is published. Those sequences are checked by NCBI
employees for aspects of quality and given an \textbf{accession number}.
For example, here an accession number for a fungal isolate that our lab
has worked with: JQ797657. You can use the NCBI program nucleotide
\textbf{BLAST} to find out more about information associated with the
isolate, in addition to getting its DNA sequence:
\url{https://blast.ncbi.nlm.nih.gov/}. Alternatively, you can use the
\texttt{read.GenBank()} function in the \texttt{ape} package to connect
to NCBI and directly get the sequence. This is pretty cool. Give it a
try.

But before your team proceeds, you need to give some thought to which
gene you want to focus on. For microorganisms like the bacteria we
worked with above, many people use the ribosomal gene (i.e., 16S rRNA).
This has many desirable features, including it is relatively long,
highly conserved, and identifies taxa with reasonable resolution. In
eukaryotes, ribosomal genes (i.e., 18S) are good for distinguishing
course taxonomic resolution (i.e.~class level), but it is not so good at
resolving genera or species. Therefore, you may need to find another
gene to work with, which might include protein-coding gene like
cytochrome oxidase (COI) which is on mitochondria and is commonly used
in molecular systematics. In plants, the ribulose-bisphosphate
carboxylase gene (\emph{rbcL}), which on the chloroplast, is commonly
used. Also, non-protein-encoding sequences like those found in
\textbf{Internal Transcribed Spacer (ITS)} regions between the small and
large subunits of of the ribosomal RNA are good for molecular
phylogenies. With your team members, do some research and identify a
good candidate gene.

After you identify an appropriate gene, download sequences and create a
properly formatted fasta file. Next, align the sequences and confirm
that you have a good alignment. Choose a substitution model and make a
tree of your choice. Based on the decisions above and the output, does
your tree jibe with what is known about the evolutionary history of your
organisms? If not, why? Is there anything you could do differently that
would improve your tree, especially with regard to future analyses done
by your team?

\begin{Shaded}
\begin{Highlighting}[]
\CommentTok{#I did my due diligence in trying to get the muscle alignment to work per your instructions over email}
\CommentTok{#but no dice.}
\CommentTok{#I didn't expect much with my computer's track record, however. }

\CommentTok{#read.aln <- read.alignment(file = ".11.PhyloTraits/wd.txt.txt", format = "fasta")}
\CommentTok{#t.DNAbin <- as.DNAbin(read.aln)}
\CommentTok{#window <- p.DNAbin[, 100:500]}
\CommentTok{#image.DNAbin(window, cex.lab = 0.50)}

\CommentTok{#seq.dist.K80 <- dist.dna(p.DNAbin, model = "K80", pairwise.deletion = FALSE)}

\CommentTok{#par(mar = c(5, 5, 2, 1) + 0.1)}
\CommentTok{#plot(seq.dist.raw, seq.dist.F84,}
 \CommentTok{#    pch = 20, col = "red", las = 1, asp = 1, xlim = c(0, 0.7), ylim = c(0, 0.7), }
  \CommentTok{#   xlab = "Raw Distance", ylab = "K80 Distance") }
\CommentTok{#abline(b = 1, a = 0, lty = 2)}
\CommentTok{#text(0.65, 0.6, "1:1")}

\CommentTok{#K80.tree <- bionj(seq.dist.K80)}
\CommentTok{#K80.outgroup <- match("Methanosarcina", K80.tree$tip.label)}
\CommentTok{#K80.rooted <- root(K80.tree, K80.outgroup, resolve.root = TRUE)}
\CommentTok{#layout(matrix(c(1,2), 1, 2), width = c(1, 1))}
\CommentTok{#par(mar = c(1, 1, 2, 0))}
\CommentTok{#par(mar = c(1, 0, 2, 1))}
\CommentTok{#plot.phylo(K80.rooted, type = "phylogram", direction = "left", show.tip.label = TRUE,}
\CommentTok{#           use.edge.length = FALSE, adj = 0.5, cex = 0.6, label.offset = 2, main = "K80")}

\CommentTok{#These sequences come from freshwater microbial data from https://mmbr.asm.org/content/mmbr/75/1/14.full.pdf}

\CommentTok{#>FJ916807.1 Uncultured actinobacterium clone CH1A11 16S ribosomal RNA gene, partial sequence}
\CommentTok{#TTTTTCGGTTGAAGATGATCTCGCGGCCTATCAGCTTGTTGGTGAGGTAATGGCTCACCAAGGCGACGAC}
\CommentTok{#GGGTAGCCGGCCTGAGAGGGCGACCGGCCACACTGGGACTGAGACACGGCCCAGACTCCTACGGGAGGCA}
\CommentTok{#GCAGTGGGGAATATTGGGCAATGGAGGAAACTCTGACCCAGCGACGCCGCGTGCGGGATGAAGGCCTTCG}
\CommentTok{#GGTTGTAAACCGCTTTCACCAGGGAAGAAGCGAAAGTGACGGTACCTGCAAAAAGAAGCACCGGCTAACT}
\CommentTok{#ATGTGCCAGCAGCCGCGGTAATACATAGGGTGCAAGCGTTGTCCGGAATTATTGGGCGTAAAGAGCTCTT}
\CommentTok{#AGGTGGTTCGTCTCGTCGGATGTGAAACTCTGGGGCTTAACCCCA}

\CommentTok{#>FJ916903.1 Uncultured Bacteroidetes bacterium clone ME7bC6 16S ribosomal RNA gene, partial sequence}
\CommentTok{#ACGGTGATCCTTCTGTACATAGCTGGTGCACGGGCTGCGTTAACACGCTATGCAACCTACCTTACATTGG}
\CommentTok{#GGGATAGCCTTTCGAAAGGGAGCATTAATACCGCATAAGACAGCTAGCTGGGCATCCAGCAGCTGTTAAA}
\CommentTok{#TGATCTATCGATGTAAGATGGGCATGCGTCCAATTAGTCAGTTGGCGAGGCAACGGCTCACCAAGACTTT}
\CommentTok{#GATTGGTAGGGGAACTGAGAGGTCAATCCCCCACACTGGCACTGAGATACGGGCCAGACTCCTACGGGAG}
\CommentTok{#GCAGCAGTAGGGAATATTGGGCAATGGACGCAAGTCTGACCCAG}

\CommentTok{#>HQ386253.1 Bacteroidetes bacterium enrichment culture clone LiUU-12-16 16S ribosomal RNA gene, partial sequence}
\CommentTok{#GGAATATTGGTCAATGGGCGCAAGCCTGAACCAGCCATGCCGCGTGCAGGAAGAATGCCCTATGGGTTGT}
\CommentTok{#AAACTGCTTTTATTTAGGAATAAACCTTTCTACGTGTAGAAAGCTGAAGGTACTGAATGAATAAGCACCG}
\CommentTok{#GCTAACTCCGTGCCAGCAGCCGCGGTAATACGGAGGGTGCAAGCGTTATCCGGAATCATTGGGTTTAAAG}
\CommentTok{#GGTCCGCAGGCGGGCGTATAAGTCAGTGGTGAAATCCTGCAGCTTAACTGCAGAACTGCCATTGATACTG}
\CommentTok{#TACGTCTTGAATTCGGTCGAAGTGGGCGGAATGTGTAGTGTAGCGGTGAAATGCATAGATATTACACAGA}
\CommentTok{#ACACCGATAGCGAAGGCTGCTCACTAGGCCTGGATTGACGCTCAGGGACGAAAGCGTGGGGATCAAACAG}
\CommentTok{#GATAGATACCCTGGTAGTCCACGC}

\CommentTok{#>HQ386631.1 Bacterium enrichment culture clone LiUU-18-315 16S ribosomal RNA gene, partial sequence}
\CommentTok{#CAGCCGCCGCGCTAATTCGAAGGGTGCGAGCGTTATTCGGATTTACTGGGCGTAAAGGGCGCGTAGGCGG}
\CommentTok{#CTTTTTAAGTCACTTGTTAAAGACCACTTAAATCGGTGGAGATGCGGGTGAAACTGGAGAGCTAGAGGGT}
\CommentTok{#AGGAGAGAGAAGTGGAATTCTTGGAGTAGCGGTAAAATGCGTAGATCTCAAGAGGAACACCGATGGCGAA}
\CommentTok{#GGCAGCTTCTTGGCCTACTACTGACGCTGAGGCGCGAAAGCGTGGGGAGCAAACAGGATTAGATACCCTG}
\CommentTok{#GTAGTCCACGCTGTAAACGATGATCACTAGATGTCAGCTTCCCTTGAGGAGGTTGGTATCGTAGCTAACG}
\CommentTok{#CGATAAGTGATCCGCCTGGGGAGTACGGTCGCAAGACTAAAACTTAAATGAATTGACGG}

\CommentTok{#>HQ530565.1 Uncultured actinobacterium clone 30LAKE01A03 16S ribosomal RNA gene, partial sequence}
\CommentTok{#CGCATTAAGCGTCCCGCCTGGGGAGTACGACCGCAAGGTTAAAACTCAAAGGAACTGACCGGGGGCCCGC}
\CommentTok{#ACAAGCAGCGGAGCATGCGGCTTAATTCGATGCAACGCGAAGAACCTTACCTAGGCTTGACATGCATTGA}
\CommentTok{#AAACTGTTAGAGATAACAGGTCCGCAAGGGCTTTGCACAGGTGGTGCATGGCTGTCGTCAGCTCGTGTCG}
\CommentTok{#TGAGATGTTGGGTTAAGTCCCGCAACGAGCGCAACCCTCGTCTTGTGTTGCCAACAGGTAATGCTGGGAA}
\CommentTok{#CTCACAAGAGACTGCCGGGGTCAACTCGGAGGAAGGTGGGGACGACGTCAAGTCATCATGCCCCTTATGT}
\CommentTok{#CTAGGGCTGCACGCATGCTACAATGGCCGATACAAAGGGCTGCGATACCGCAAGGTGGAGCGAATCCCAT}
\CommentTok{#AAAGTCGGTCTCAGTTCGGATTGAGGTCTGCAACTCGACCTCATGAAGTTGGAGTTGCTAGTAATCGCGG}
\CommentTok{#ATCAGCAACGCCGCGGTGAATACGTTCCCGGGCCTTGTACACACCGCCCGTCACGTCACGAAAGTCGGTA}
\CommentTok{#ACACCCAAAACCAGTGTTCCAACCGTAAGGGGGAAGCTG}

\CommentTok{#>HQ532908.1 Uncultured actinobacterium clone TB1H9 16S ribosomal RNA gene, partial sequence}
\CommentTok{#TTCGACGCAACGCGAAGAACCTTACCAAGGCTCGACATATACCGAAAAGCAGCAGAGATGTTGTGTCCGC}
\CommentTok{#AAGGGCGGTATACAGGTGGTGCATGGTTGTCGTCAGCTCGTGTCGTGAGATGTTGGGTTAAGTCCCGCAA}
\CommentTok{#CGGGCGCAACCCTCGTTCTGTGTTGCCAGCATTTAGTTGGGGGCTCACAGGAGACTGCCGTGGTCAACAC}
\CommentTok{#GGAGGAAGGTGGGGATGACGTCAAATCATCATGCCCCTTATGTCTTGGGCTGCACGCATGCTACAATGGC}
\CommentTok{#AGGTACAAAGGGCTGCGATACCGTAAGGTGGAGCGAATCCCAAAAAGTCTGTCTCAGTTCGGATTGAGGT}
\CommentTok{#CTGCAACTCGACCTCATGAAGTCGGAGTTGCTAGTAATCGTAGATCAGCAACGCTACGGTGAATACGTTC}
\CommentTok{#CCGGGGCTTGTACACACCGCCCGTCACGTCACGAAAGTCGGTAACACCCAAAGTCAGTGGCCCAACCGCA}
\CommentTok{#AGGA}

\CommentTok{#>FJ916875.1 Uncultured Bacteroidetes bacterium clone MI1A7 16S ribosomal RNA gene, partial sequence}
\CommentTok{#TCGAGGGGTAGAGTAAGCTTGCTTACTTGAGACCGGCGCACGGGTGCGTAACGCGTATGCAATCTACCTT}
\CommentTok{#ATACAGGGGAATAGCCCAGAGAAATTTGGATTAATGCCCCATGGTATTATAGAGTGGCATCACTTTATAA}
\CommentTok{#TTAAAGTTCCAACGGTATAAGATGAGCATGCGTCCCATTAGTTAGTTGGTAAGGTAACGGCTTACCAAGA}
\CommentTok{#CGATGATGGGTAGGGGTCCTGAGAGGGAGATCCCCCACACTGGTACTGAGACACGGACCA}

\CommentTok{#>HQ386400.1 Bacterium enrichment culture clone LiUU-15-729 16S ribosomal RNA gene, partial sequence}
\CommentTok{#CAGCCGCCGCGGTAATACAGAGGATGCAAGCGTTATCCGGAATCACTGGGCATAAAGCGTCTGTAGGTTG}
\CommentTok{#CCTACCAAGTCTGCTGTTAAAGATCAGGGCCTAACCCTGGGAAAGCAGTGGAAACTAGTAGGCTTGAGTG}
\CommentTok{#TGGTAGAGGTAGAGGGAATTCCTGGTGTAGCGGTGAAATGCGTAGATATTAGGAAGAACACCAATGGCGA}
\CommentTok{#AAGCACTCTACTGGGCCATAACTGACACTGAGAGACGACAGCTAGGGGAGCAAATGGGATTAGATACCCC}
\CommentTok{#AGTAGTCCTAGCCGTAAACGATGGATACTAGGTGTTGCACGTATTAACCCGTGCAGTATCGTAGCTAAGG}
\CommentTok{#CGTTAAGTATCCCGCCTGGGAAGTACGCTCGCAAGAGTGAAACTTAAATGAATTGACGG}

\CommentTok{#>HQ530755.1 Uncultured actinobacterium clone 30LAKE31D05 16S ribosomal RNA gene, partial sequence}
\CommentTok{#CGGAGCATGCGGCTTAATTCGACGCAACGCGAAGAACCTTACCAAGGCTTGACATATACAGGAATATGGC}
\CommentTok{#AGAGATGTCATAGCCGCAAGGTCTGTATACAGGTGGTGCATGGTTGTCGTCAGCTCGTGTCGTGAGATGT}
\CommentTok{#TGGGTTAAGTCCCGCAACGAGCGCAACCCTCGTTCTGTGTTGCCAGCATTTAGTTGGGGACTCACAGGAG}
\CommentTok{#ACTGCCGGGGTTAACTCGGAGGAAGGTGGGGATGACGTCAAATCATCATGCCCCTTATGTCTTGGGCTGC}
\CommentTok{#ACGCATGCTACAATGGCTGGTACAAACGGCTGCGATACCGCAAGGTGGAGCGAATCCGATAAAGCCAGTC}
\CommentTok{#TCAGTTCGGATTGGGGTCTGCAACTCGACCCCATGAAGTCGGAGTTGCTAGTAATCGTAGATCAGCAACG}
\CommentTok{#CTACGGTGAATACGTTCCCGGGGCTTGTACACACCGCCCGTCACATCACGAAAGTCGGAACACCCAAAGT}
\CommentTok{#CAGT}

\CommentTok{#>FJ916900.1 Uncultured beta proteobacterium clone LT1bG1 16S ribosomal RNA gene, partial sequence}
\CommentTok{#AACGCGCTGAGTAATACATCGTGAACGTACCTTATCGTGGGGGATAACGCAGCGAAAGTTGCGCTAACTA}
\CommentTok{#CCGCATACGCCCTGAGGGGGAAAGCGGGGGACCGTAAGGCCTCGCGCGATTAGAGCGGCCGATGTCTGAT}
\CommentTok{#TAGCTTGTTGGTGAGGTAAAAGCTCACCAAGGCGATGATCAGTAGCTGGTCTGAGAGGACGATCATCCAC}
\CommentTok{#ACTGGGACTGAGACACGGCCCAGACTCCTACGGGAGGCAGCAGTGGGGAATTTTGGACAATGGGGGCAAC}
\CommentTok{#CCTGATCCAGCAATGCCT}

\CommentTok{#>FJ917000.1 Batocarpus costaricensis voucher GW1463 18S ribosomal RNA gene, partial sequence; internal transcribed spacer 1, 5.8S ribosomal RNA gene, and #internal transcribed spacer 2, complete sequence; and 28S ribosomal RNA gene, partial sequence}
\CommentTok{#MTKCSAWWGATATTGTTGAAACCTGCCCAGCAGAAAGACCCGTGAACACGTTACAACACTCGGGGGGGCG}
\CommentTok{#AGGGGCGCCGCGCGCGCCTTGATTCCCCCCACGCCGAGTGCGCGTGGCTTTGCCCCGCGTCCCCGGACAC}
\CommentTok{#TAACCAACCCCGGCGCGGAATGCGTCAAGGAAACATAAACAAACGAGCTTCTGCTGCAGCCCCGGACTCG}
\CommentTok{#GTGCTTGCTGCGGCGGACGTGTCGTGTTTCGTTTAAGTCTTAAAACGACTCTCGGCAACGGATATCTCGG}
\CommentTok{#CTCTCGCATCGATRAAGAACGTAGCGAAATGCGATACTTGGTGTGAATTGCAGAATCCCGTGAACCATCG}
\CommentTok{#AGTCTTTGAACGCAAGTTGCGCCCGAAGCCATCCGGCCGAGGGCACGTCTGCCTGGGCGTCACACACCGT}
\CommentTok{#TGCCCCCCCTAAATCCCTTTGGCGCCGTCCATATGGTGCATGGGGACTACGGGGGGCGGATGATGGCCTC}
\CommentTok{#CCGTGGGCCTTGACCCGCGGCTGGTCTAAATTTGAGTCCCCGGTCACGGTTGCCGTGGCAATAGGTGGTC}
\CommentTok{#GTCGAACATTCGGTGCCCCGCCATGTGCTCTGGACAGAAAGCATCGGGAGACTTTACAAACTCGACCCCG}
\CommentTok{#ACGCACCCCACGCGGGTGCTTCCAACGCGAACCCAGGTCAGCG}

\CommentTok{#>FJ916950.1 Physalia sp. DRP-2009 isolate Foxton3 cytochrome oxidase subunit I (COI) gene, partial cds; mitochondrial}
\CommentTok{#CCGGGAGCTCTCATATTAAATATAGTGGTTATAAAGTTAATGGCACCCATGATTGATGAGGCACCTGCAC}
\CommentTok{#AGTGTAAACTGAAGATAGCCATATCAACTGACCCTCCGGAATGAGTTTGGGGGCCAGACAAAGGAGGGTA}
\CommentTok{#AACAGTTCAACCAGTACCTGCACCTTGTTCTATTAGAGATGACCCTAGTAATAATAATAGAGCAGGGGGC}
\CommentTok{#AGTAACCAAAAACTTAGGTTATTCAACCTAGGAAAGGCCATATCCGGAGCACCTATAAACAGGGGTACGA}
\CommentTok{#ACCAGTTACCAAATCCTCCAATTAGAACTGGCATTACAAGGAAAAAGATCATAACAAATGCATGAGCAGT}
\CommentTok{#TACTATAACATTATAGAGATGATCATCTCCAAACATGGTACCAGGTCCTGACAATTCCAATCTGA}
\end{Highlighting}
\end{Shaded}


\end{document}
